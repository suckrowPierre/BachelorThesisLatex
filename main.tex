% !TeX spellcheck = en-US
% !TeX encoding = utf8
% !TeX program = pdflatex
% !BIB program = biber
% -*- coding:utf-8 mod:LaTeX -*-


% vv  scroll down to line 200 for content  vv


\let\ifdeutsch\iftrue
\let\ifenglish\iffalse


\input{pre-documentclass}
\documentclass[
  %
  %ngerman, %%% Add if you write in German.
  %
  % fontsize=11pt is the standard
  a4paper,  % Standard format - only KOMAScript uses paper=a4 - https://tex.stackexchange.com/a/61044/9075
  twoside,  % we are optimizing for both screen and two-side printing. So the page numbers will jump, but the content is configured to stay in the middle (by using the geometry package)
  bibliography=totoc,
  %               idxtotoc,   %Index ins Inhaltsverzeichnis
  %               liststotoc, %List of X ins Inhaltsverzeichnis, mit liststotocnumbered werden die Abbildungsverzeichnisse nummeriert
  headsepline,
  cleardoublepage=empty,
  parskip=half,
  %               draft    % um zu sehen, wo noch nachgebessert werden muss - wichtig, da Bindungskorrektur mit drin
  draft=false
]{scrbook}
\input{config}


\usepackage[
  title={Text-zu-spielbarem-Klang: Synthesizer basierend auf Latent-Diffusion-Technologie}, % Do not forget to capitalize your title correctly, you may use the following page to help you: https://capitalizemytitle.com/
  author={Pierre-Louis Wolgang Léon Suckrow},
  orcid=0000-0000-0000-0000, % get your own ORCID via https://orcid.org/
  email={P.Suckrow@campus.lmu.de},
  emailsecond= {suckrowpierre@gmail.com},
  type=bachelor,
  institute={Institut für Informatik}, % or other institute names - or just a plain string using {Demo\\Demo...}
  course={Informatik},
  examiner={Prof.\ Dr.\ Sylvia Rothe},
  supervisor={Dipl.-Inf.\ Christoph Weber},
  startdate={Mai 31, 2023},
  enddate={August 19, 2023},
  % Falls keine Lizenz gewünscht wird bitte auf "none" setzen
  % Die Lizenz erlaubt es zu nichtkommerziellen Zwecken die Arbeit zu
  % vervielfältigen und Kopien zu machen. Dabei muss aber immer der Autor
  % angegeben werden. Eine kommerzielle Verwertung ist für den Autor
  % weiter möglich.
  copyright=ccbysa, % ccbysa, ccbynosa, cc0, none
  language=german
]{lmu-thesis-cover}

\input{acronyms}

\makeindex

\begin{document}

%tex4ht-Konvertierung verschönern
\iftex4ht
  % tell tex4ht to create picures also for formulas starting with '$'
  % WARNING: a tex4ht run now takes forever!
  \Configure{$}{\PicMath}{\EndPicMath}{}
  %$ % <- syntax highlighting fix for emacs
  \Css{body {text-align:justify;}}

  %conversion of .pdf to .png
  \Configure{graphics*}
  {pdf}
  {\Needs{"convert \csname Gin@base\endcsname.pdf
      \csname Gin@base\endcsname.png"}%
    \Picture[pict]{\csname Gin@base\endcsname.png}%
  }
\fi

%\VerbatimFootnotes %verbatim text in Fußnoten erlauben. Geht normalerweise nicht.

\input{commands}
\pagenumbering{arabic}
\Coverpage
\Copyright
%Eigener Seitenstil fuer die Kurzfassung und das Inhaltsverzeichnis
\deftriplepagestyle{preamble}{}{}{}{}{}{\pagemark}
%Doku zu deftriplepagestyle: scrguide.pdf
\pagestyle{preamble}
\renewcommand*{\chapterpagestyle}{preamble}



%Kurzfassung / abstract
%auch im Stil vom Inhaltsverzeichnis
\section*{Kurzfassung}

\todo{Short summary of the thesis. Here, the following questions should be answered:}
\todo{What is the specific problem addressed?}
\todo{What have you done?}
\todo{What did you find out?}
\todo{What are the implications on a larger scale?}
\todo{Should be around 0.5 pages. Not longer than 1 page.}

\cleardoublepage

\section*{Abstract}

\todo{Short summary of the thesis. Here, the following questions should be answered:}
\todo{What is the specific problem addressed?}
\todo{What have you done?}
\todo{What did you find out?}
\todo{What are the implications on a larger scale?}
\todo{Should be around 0.5 pages. Not longer than 1 page.}

\cleardoublepage


% BEGIN: Verzeichnisse

\iftex4ht
\else
  \microtypesetup{protrusion=false}
\fi

%%%
% Literaturverzeichnis ins TOC mit aufnehmen, aber nur wenn nichts anderes mehr hilft!
% \addcontentsline{toc}{chapter}{Literaturverzeichnis}
%
% oder zB
%\addcontentsline{toc}{section}{Abkürzungsverzeichnis}
%
%%%

%Produce table of contents
%
%In case you have trouble with headings reaching into the page numbers, enable the following three lines.
%Hint by http://golatex.de/inhaltsverzeichnis-schreibt-ueber-rand-t3106.html
%
%\makeatletter
%\renewcommand{\@pnumwidth}{2em}
%\makeatother
%
\tableofcontents

% Bei einem ungünstigen Seitenumbruch im Inhaltsverzeichnis, kann dieser mit
% \addtocontents{toc}{\protect\newpage}
% an der passenden Stelle im Fließtext erzwungen werden.

\listoffigures
\listoftables

% Control List of Listings
\let\iflistings\iffalse
%Wird nur bei Verwendung von der lstlisting-Umgebung mit dem "caption"-Parameter benoetigt
%\lstlistoflistings
%ansonsten:
\iflistings
  \ifdeutsch
    \listof{Listing}{Verzeichnis der Listings}
  \else
    \listof{Listing}{List of Listings}
  \fi
\fi

% Control List of Algorithms
\let\ifalgorithms\iffalse
\ifalgorithms
  %mittels \newfloat wurde die Algorithmus-Gleitumgebung definiert.
  %Mit folgendem Befehl werden alle floats dieses Typs ausgegeben
  \ifdeutsch
    \listof{Algorithmus}{Verzeichnis der Algorithmen}
  \else
    \listof{Algorithmus}{List of Algorithms}
  \fi
  %\listofalgorithms %Ist nur für Algorithmen, die mittels \begin{algorithm} umschlossen werden, nötig
\fi

% Control Glossary
\let\ifglossary\iffalse
\ifglossary
  \printnoidxglossaries
\fi

\iftex4ht
\else
  %Optischen Randausgleich und Grauwertkorrektur wieder aktivieren
  \microtypesetup{protrusion=true}
\fi

% END: Verzeichnisse


% Headline and footline
\renewcommand*{\chapterpagestyle}{scrplain}
\pagestyle{scrheadings}
\pagestyle{scrheadings}
\ihead[]{}
\chead[]{}
\ohead[]{\headmark}
\cfoot[]{}
\ofoot[\usekomafont{pagenumber}\thepage]{\usekomafont{pagenumber}\thepage}
\ifoot[]{}


%% vv  scroll down for content  vv %%































%%%%%%%%%%%%%%%%%%%%%%%%%%%%%%%%%%%%%%%%%%%%%%%%%%%%%%%%%%%%%%%%%%%%%%%%%%%%%%
%
% Main content starts here
%
%%%%%%%%%%%%%%%%%%%%%%%%%%%%%%%%%%%%%%%%%%%%%%%%%%%%%%%%%%%%%%%%%%%%%%%%%%%%%%


\chapter{Einleitung}
\label{sec:introduction}

\glqq Es gibt keine theoretischen Grenzen für den Computer als Quelle für musikalische Klänge, im Gegensatz zu herkömmlichen Instrumenten\grqq, \cite{mathews_digital_1963}. Mit diesen wegweisenden Worten eröffnete Max Vernon Mathews im Jahr 1963 seinen Artikel "The Digital Computer as a Musical Instrument" und legte durch seine Arbeit an den Bell Laboratories den Grundstein für die heutige Musikproduktion und Klangsynthese am Computer \cite{mathews_music_2004}.

In Anlehnung an die Gedanken von M. V. Mathews fokussiert sich diese Arbeit auf die Implementierung eines Synthesizers mittels neuer Latenter Diffusionstechnologie. Das Ziel ist es, einem Nutzer durch die Nutzung dieses digitalen Instruments die Generierung von Klängen nach seinen individuellen Vorstellungen zu ermöglichen, die anschließend manipuliert und spielbar sein sollen. Weiterhin soll das Instrument nahtlos in den modernen Musikproduktionsprozess integriert werden können und Musikern die Erschließung neuer Klangwelten ermöglichen. Durch die Erzeugung neuer, gezielter Klänge können zudem rechtliche Probleme vermieden werden, die durch die Verwendung von Samples entstehen könnten.

Verschiedene ML-Modelle wie (Beispiele) beschäftigen sich mit der Erzeugung von Musikalischen 
This is a typical human-computer interaction thesis structure for an introduction which is structured in four paragraphs as follows:
% First Paragraph
% CORE MESSAGE OF THIS PARAGRAPH:
\todo{P1.1. What is the large scope of the problem?}
\todo{P1.2. What is the specific problem?}

% Second Paragraph
% CORE MESSAGE OF THIS PARAGRAPH:
\todo{P2.1. The second paragraph should be about what have others been doing}
\todo{P2.2. Why is the problem important? Why was this work carried out?}

% Third Paragraph
% CORE MESSAGE OF THIS PARAGRAPH:
\todo{P3.1. What have you done?}
\todo{P3.2. What is new about your work?}

% Fourth paragraph
% CORE MESSAGE OF THIS PARAGRAPH:
\todo{P4.1. What did you find out? What are the concrete results?}
\todo{P4.2. What are the implications? What does this mean for the bigger picture?}

LaTeX hints are provided in \autoref{chap:latexhints}.

\chapter{Related Work}

Describe relevant scientific literature related to your work.

\chapter{Ziel}

\chapter{Methoden}

\section{Klangsynthese und Musikproduktion}

\subsection{Mathematische und physikalische Modellierung von Musik und Klang}
\glqq Musik ist eine Kunstform und kulturelle Aktivität, deren Medium der in Zeit organisierte Klang ist\grqq \, \cite{tsuji_physics_2021}. Im Gegensatz zu Rauschen weisen die Klänge, die Musik aufbauen, Strukturen und Zusammenhänge auf, welche für das menschliche Gehör als angenehm wahrgenommen werden \cite{parker_good_2009}.  

\emph{Klang} stellt ein intrinsisches Zusammenspiel aus physikalischen und perzeptiven Elementen dar. Auf physikalischer Ebene handelt es sich bei Klang um eine durch einen schwingenden Körper erzeugte Welle, die sich von einem Ort zum anderen propagiert. Diese Welle besteht aus einem Grundton und mehreren resonierenden Einzeltönen. Der resultierende Klang besitzt eine Vielzahl von Obertönen, die die charakteristischen klanglichen Eigenschaften oder \emph{Klangfarben} hervorbringen \cite{tsuji_physics_2021, parker_good_2009}.

Diese verschiedenen \emph{Töne} sind periodische Schwingungen, definiert durch eine \emph{Tonhöhe/Frequenz} $\omega$ (in $Hz$), welche als die Anzahl der Kompressionen an einem bestimmten Punkt pro Sekunde interpretiert werden kann. Der Kehrwert der Frequenz wird als \emph{Periode} $T=\frac{1}{\omega}$ (in $s$) bezeichnet und beschreibt die Zeit, die eine Kompression benötigt, um zwei identische Punkte zu passieren. Unsere Wahrnehmung lässt Töne mit niedriger Frequenz tief und dumpf klingen, während hohe Frequenzen als leicht, schwebend und durchdringend wahrgenommen werden. Die \emph{Amplitude} der Schwingung beschreibt die transportierte Energie und somit die Lautstärke eines Tones. Aufgrund des Abstandsgesetzes wird diese von einer Schallquelle über die logarithmische \emph{Dezibel-Skala} (dB) angegeben.  \cite{tsuji_physics_2021, parker_good_2009}

Die Untersuchung der Struktur eines Klangs befasst sich mit den Relationen und Zusammenhängen der verschiedenen Töne, aus denen der Klang besteht. Die einfachste Form eines Tones ist der Sinuston, dessen Schwingung durch eine Sinuskurve dargestellt wird. Gemäß dem Fouriertheorem setzt sich jeder andere Ton aus verschiedenen Sinustönen zusammen, wobei die tiefste dominante Frequenz als \emph{Grundton} und die höheren Frequenzen als \emph{Obertöne} bezeichnet werden. Die Obertöne unterscheiden sich in Frequenz, Amplitude und zeitlichem Auf- und Abbau. Wenn die Frequenz eines Obertones ein ganzzahliges Vielfaches des Grundtones ist, wird dieser als \emph{harmonischer} Oberton bezeichnet. Im Falle, dass der Oberton kein ganzzahliges Vielfaches des Grundtones ist, spricht man von einem \emph{inharmonischen} Oberton. Daher erzeugen unterschiedliche Instrumente, die die gleiche Note spielen, die gleiche Grundschwingung, weisen jedoch unterschiedliche harmonische und inharmonische Obertöne auf. \cite{parker_good_2009, white_physics_2014, ruschkowski_elektronische_2019}

Die Struktur eines Klangs kann durch die Amplitudenfaktoren und zugehörigen Frequenzen der einzelnen Töne beschrieben und in einem \emph{Frequenzspektrum} visualisiert werden. Dies erlaubt die Bestimmung, welche Frequenzen oder Frequenzbereiche im Signal besonders stark enthalten sind und ermöglicht eine Darstellung der Klangfarbe bzw. der Charakteristik. Mit einer \emph{Fourier-Transformation} kann für jedes beliebige Signal das entsprechende Spektrum ermittelt und durch die \emph{Inverse Fourier-Transformation} das Signal eines Spektrums berechnet werden \cite{raffaseder_audiodesign_2010}. Zur Berechnung einer \emph{Diskreten Fourier-Transformation (DFT)} hat sich die \emph{Fast Fourier-Transformation (FFT)} als effizienter Algorithmus etabliert \cite{heideman_gauss_1985}. Ein \emph{Spektrogramm} ermöglicht die Betrachtung eines Signals sowohl im Zeit- als auch im Frequenzbereich, allerdings nicht in beliebiger Genauigkeit. Da hier das Signal in kurze zeitliche Abschnitte unterteilt und für diese das Spektrum berechnet wird, verlieren längere Abschnitte Informationen über die zeitliche Auflösung, weisen jedoch eine bessere Frequenzauflösung auf. Kurze Abschnitte hingegen besitzen eine bessere zeitliche Auflösung, lösen jedoch weniger genau im Frequenzbereich auf. \cite{raffaseder_audiodesign_2010}

Ein Klang wird als \emph{Rauschen} bezeichnet, wenn er ein kontinuierliches Frequenzspektrum aufweist, das viele Arten von Geräuschen aus unterschiedlichen Quellen umfasst.  
\cite{tsuji_physics_2021}

Der durch das menschliche Gehör erfassbare Frequenzbereich wird in drei Bereiche (\emph{Bässe, Mitten, Höhen}) unterteilt. Bässe sind in einem Bereich von 20$Hz$ bis 250$Hz$ auffindbar, während tiefe Mitten den Bereich von 250$Hz$ bis 2000$Hz$ umschließen. Hohen Mitten erstrecken sich von 2$kHZ$ bis 4$kHz$. Die Höhen liegen oberhalb von 4$kHz$. \cite{raffaseder_audiodesign_2010}
\subsection{Digitale Audiorepräsentation und Sampling}. 
\subsection{Synthetisierung und Klangformung}


\section{Synthetisierung mittels Diffusion}
\subsection{Latente Diffsuion}
\subsection{AudioLDM}
\subsection{Finetuning}
\section{Implementierung des Neuronalen Synthesizers}
\subsection{Audioprogrammierung}
\subsection{JUCE Framework}
\glqq\emph{JUCE} ist das am häufigsten verwendete Framework für die Entwicklung von Audioanwendungen und -Plugins. Es handelt sich dabei um eine Open-Source-C++-Codebasis, die zur Erstellung eigenständiger Software auf Windows, macOS, Linux, iOS und Android sowie VST-, VST3-, AU-, AUv3-, AAX- und LV2-Plugins verwendet werden kann.\grqq \cite{noauthor_juce_nodate}

Es bietet eine Abstraktion für die Verarbeitung von Audiosamples und MIDI von den nativen Audiogeräten auf jeder Plattform oder einer Host-DAW. Mit der Bibliothek von \emph{digitalen Signalverarbeitungs-(DSP)-Bausteinen}, die JUCE bereitstellt, können unterschiedliche Audioeffekte, Filter, Instrumente und Generatoren schnell prototypisiert und eingesetzt werden. \cite{noauthor_juce_nodate} So umfasst wa eine breite Palette von Klassen, die häufig auftretende Probleme bei der Entwicklung von Audioprojekten lösen. Dazu gehören die Behandlung von Grafiken, Sound, Benutzerinteraktion und Netzwerken. \cite{robinson_getting_2013}

In dieser Arbeit wird das JUCE-Framework mittels \emph{CMake}, \glqq eine Open-Source-, plattformübergreifende Werkzeugfamilie, die zur Erstellung, zum Testen und zum Verpacken von Software entwickelt wurde\grqq \cite{noauthor_cmake_nodate} benutzt, um die durch das Diffusionsnetz erstellten Klänge spielbar und manipulierbar zu gestalten. 
\subsection{ONNX}
\subsection{Design}

\chapter{Ergebnisse}

\chapter{Diskussion}

\chapter{Schlussfolgerung}
\label{sec:conclusion}


\printbibliography

All links were last followed on \today{}.

\appendix
\input{latexhints/latexhints-english}

\pagestyle{empty}
\renewcommand*{\chapterpagestyle}{empty}
\Affirmation
\end{document}
