% !TeX spellcheck = en-US
% !TeX encoding = utf8
% !TeX program = pdflatex
% !BIB program = biber
% -*- coding:utf-8 mod:LaTeX -*-


% vv  scroll down to line 200 for content  vv


\let\ifdeutsch\iftrue
\let\ifenglish\iffalse


\input{pre-documentclass}
\documentclass[
  %
  %ngerman, %%% Add if you write in German.
  %
  % fontsize=11pt is the standard
  a4paper,  % Standard format - only KOMAScript uses paper=a4 - https://tex.stackexchange.com/a/61044/9075
  twoside,  % we are optimizing for both screen and two-side printing. So the page numbers will jump, but the content is configured to stay in the middle (by using the geometry package)
  bibliography=totoc,
  %               idxtotoc,   %Index ins Inhaltsverzeichnis
  %               liststotoc, %List of X ins Inhaltsverzeichnis, mit liststotocnumbered werden die Abbildungsverzeichnisse nummeriert
  headsepline,
  cleardoublepage=empty,
  parskip=half,
  %               draft    % um zu sehen, wo noch nachgebessert werden muss - wichtig, da Bindungskorrektur mit drin
  draft=false
]{scrbook}
\input{config}


\usepackage[
  title={Text-zu-spielbarem-Klang: Synthesizer basierend auf Latent-Diffusion-Technologie}, % Do not forget to capitalize your title correctly, you may use the following page to help you: https://capitalizemytitle.com/
  author={Pierre-Louis Wolgang Léon Suckrow},
  orcid=0000-0000-0000-0000, % get your own ORCID via https://orcid.org/
  email={P.Suckrow@campus.lmu.de},
  emailsecond= {suckrowpierre@gmail.com},
  type=bachelor,
  institute={Institut für Informatik}, % or other institute names - or just a plain string using {Demo\\Demo...}
  course={Informatik},
  examiner={Prof.\ Dr.\ Sylvia Rothe},
  supervisor={Dipl.-Inf.\ Christoph Weber},
  startdate={Mai 31, 2023},
  enddate={August 9, 2023},
  % Falls keine Lizenz gewünscht wird bitte auf "none" setzen
  % Die Lizenz erlaubt es zu nichtkommerziellen Zwecken die Arbeit zu
  % vervielfältigen und Kopien zu machen. Dabei muss aber immer der Autor
  % angegeben werden. Eine kommerzielle Verwertung ist für den Autor
  % weiter möglich.
  copyright=ccbysa, % ccbysa, ccbynosa, cc0, none
  language=german
]{lmu-thesis-cover}

\input{acronyms}

\makeindex

\begin{document}

%tex4ht-Konvertierung verschönern
\iftex4ht
  % tell tex4ht to create picures also for formulas starting with '$'
  % WARNING: a tex4ht run now takes forever!
  \Configure{$}{\PicMath}{\EndPicMath}{}
  %$ % <- syntax highlighting fix for emacs
  \Css{body {text-align:justify;}}

  %conversion of .pdf to .png
  \Configure{graphics*}
  {pdf}
  {\Needs{"convert \csname Gin@base\endcsname.pdf
      \csname Gin@base\endcsname.png"}%
    \Picture[pict]{\csname Gin@base\endcsname.png}%
  }
\fi

%\VerbatimFootnotes %verbatim text in Fußnoten erlauben. Geht normalerweise nicht.

\input{commands}
\pagenumbering{arabic}
\Coverpage
\Copyright
%Eigener Seitenstil fuer die Kurzfassung und das Inhaltsverzeichnis
\deftriplepagestyle{preamble}{}{}{}{}{}{\pagemark}
%Doku zu deftriplepagestyle: scrguide.pdf
\pagestyle{preamble}
\renewcommand*{\chapterpagestyle}{preamble}



%Kurzfassung / abstract
%auch im Stil vom Inhaltsverzeichnis
\section*{Kurzfassung}

\todo{Short summary of the thesis. Here, the following questions should be answered:}
\todo{What is the specific problem addressed?}
\todo{What have you done?}
\todo{What did you find out?}
\todo{What are the implications on a larger scale?}
\todo{Should be around 0.5 pages. Not longer than 1 page.}

\cleardoublepage

\section*{Abstract}

\todo{Short summary of the thesis. Here, the following questions should be answered:}
\todo{What is the specific problem addressed?}
\todo{What have you done?}
\todo{What did you find out?}
\todo{What are the implications on a larger scale?}
\todo{Should be around 0.5 pages. Not longer than 1 page.}

\cleardoublepage


% BEGIN: Verzeichnisse

\iftex4ht
\else
  \microtypesetup{protrusion=false}
\fi

%%%
% Literaturverzeichnis ins TOC mit aufnehmen, aber nur wenn nichts anderes mehr hilft!
% \addcontentsline{toc}{chapter}{Literaturverzeichnis}
%
% oder zB
%\addcontentsline{toc}{section}{Abkürzungsverzeichnis}
%
%%%

%Produce table of contents
%
%In case you have trouble with headings reaching into the page numbers, enable the following three lines.
%Hint by http://golatex.de/inhaltsverzeichnis-schreibt-ueber-rand-t3106.html
%
%\makeatletter
%\renewcommand{\@pnumwidth}{2em}
%\makeatother
%
\tableofcontents

% Bei einem ungünstigen Seitenumbruch im Inhaltsverzeichnis, kann dieser mit
% \addtocontents{toc}{\protect\newpage}
% an der passenden Stelle im Fließtext erzwungen werden.

\listoffigures
\listoftables

% Control List of Listings
\let\iflistings\iffalse
%Wird nur bei Verwendung von der lstlisting-Umgebung mit dem "caption"-Parameter benoetigt
%\lstlistoflistings
%ansonsten:
\iflistings
  \ifdeutsch
    \listof{Listing}{Verzeichnis der Listings}
  \else
    \listof{Listing}{List of Listings}
  \fi
\fi

% Control List of Algorithms
\let\ifalgorithms\iffalse
\ifalgorithms
  %mittels \newfloat wurde die Algorithmus-Gleitumgebung definiert.
  %Mit folgendem Befehl werden alle floats dieses Typs ausgegeben
  \ifdeutsch
    \listof{Algorithmus}{Verzeichnis der Algorithmen}
  \else
    \listof{Algorithmus}{List of Algorithms}
  \fi
  %\listofalgorithms %Ist nur für Algorithmen, die mittels \begin{algorithm} umschlossen werden, nötig
\fi

% Control Glossary
\let\ifglossary\iffalse
\ifglossary
  \printnoidxglossaries
\fi

\iftex4ht
\else
  %Optischen Randausgleich und Grauwertkorrektur wieder aktivieren
  \microtypesetup{protrusion=true}
\fi

% END: Verzeichnisse


% Headline and footline
\renewcommand*{\chapterpagestyle}{scrplain}
\pagestyle{scrheadings}
\pagestyle{scrheadings}
\ihead[]{}
\chead[]{}
\ohead[]{\headmark}
\cfoot[]{}
\ofoot[\usekomafont{pagenumber}\thepage]{\usekomafont{pagenumber}\thepage}
\ifoot[]{}


%% vv  scroll down for content  vv %%































%%%%%%%%%%%%%%%%%%%%%%%%%%%%%%%%%%%%%%%%%%%%%%%%%%%%%%%%%%%%%%%%%%%%%%%%%%%%%%
%
% Main content starts here
%
%%%%%%%%%%%%%%%%%%%%%%%%%%%%%%%%%%%%%%%%%%%%%%%%%%%%%%%%%%%%%%%%%%%%%%%%%%%%%%


\chapter{Einleitung}
\label{sec:introduction}

\glqq Es gibt keine theoretischen Grenzen für den Computer als Quelle für musikalische Klänge, im Gegensatz zu herkömmlichen Instrumenten \grqq\, 
\cite{mathews_digital_1963} Mit diesen Worten leitete Max Vernon Mathews 1963 seinen Artikel ``The Digital Computer as a Musical Instrument'' ein und legte mit seiner Arbeit an den Bell Laboratories die Grundlagen für die heutige Musikproduktion und Klangerzeugen am Computer. \cite{mathews_music_2004}. Getreu den Gedanken M. V. Mathews soll in dieser Arbeit das Implementieren eines Synthesizer mittels neuer Latenter Diffusions Technologie untersucht werden.  

This is a typical human-computer interaction thesis structure for an introduction which is structured in four paragraphs as follows:
% First Paragraph
% CORE MESSAGE OF THIS PARAGRAPH:
\todo{P1.1. What is the large scope of the problem?}
\todo{P1.2. What is the specific problem?}

% Second Paragraph
% CORE MESSAGE OF THIS PARAGRAPH:
\todo{P2.1. The second paragraph should be about what have others been doing}
\todo{P2.2. Why is the problem important? Why was this work carried out?}

% Third Paragraph
% CORE MESSAGE OF THIS PARAGRAPH:
\todo{P3.1. What have you done?}
\todo{P3.2. What is new about your work?}

% Fourth paragraph
% CORE MESSAGE OF THIS PARAGRAPH:
\todo{P4.1. What did you find out? What are the concrete results?}
\todo{P4.2. What are the implications? What does this mean for the bigger picture?}

LaTeX hints are provided in \autoref{chap:latexhints}.

\chapter{Related Work}

Describe relevant scientific literature related to your work.

\chapter{Ziel}

\chapter{Methoden}

\section{Klangsynthese und Musikproduktion}

\subsection{Mathematische und physikalische Modellierung von Musik und Klang}
\glqq Musik ist eine Kunstform und kulturelle Aktivität, deren Medium der in Zeit organisierte Klang ist\grqq \, \cite{tsuji_physics_2021}. Im Gegensatz zu Rauschen weisen die Klänge, die Musik aufbauen, Strukturen und Zusammenhänge auf, welche für das menschliche Gehör als angenehm empfunden werden \cite{parker_good_2009}.  


\emph{Klang} ist ein intrinsisches Zusammenspiel aus physikalischen und perzeptiven Elementen. Physikalisch beschrieben ist Klang eine Welle, ausgelöst durch einen schwingenden Korpus, welche von einem Ort zum anderen propagiert. Diese Welle setzt sich aus einem Grundton und mehreren resonierenden Einzeltönen zusammen. Der daraus resultierende Klang weist eine Reihe von Obertönen auf und erzeugt die charakteristischen klanglichen Eigenschaften oder \emph{Klangfarben} \cite{tsuji_physics_2021, parker_good_2009}.

Diese verschiedenen \emph{Töne} sind periodische Schwingungen definiert durch eine \emph{Tonhöhe/Frequenz} $\omega$ (in $Hz$), welche als die Anzahl an Kompressionen eines bestimmen Punktes pro Sekunde gesehen werden kann. Der Kehrwert der Frequenz wird als \emph{Periode} $T=\frac{1}{\omega}$ (in $s$) bezeichnet und beschreibt die Zeit, welche eine Kompression braucht um zwei identische Punkte zu passieren. Unsere Wahrnehmung, lassen Töne mit einer niedrigen Frequenz tief und dumpf klingen, während hohe Frequenzen leicht, schwebend, durchdringend klingen. Die \emph{Amplitude} der Schwingung berschreibt die transportierte Energie und somit die Lautstärke eines Tones. Diese wird auf Grund des Abstandsgesetz von einer Tonquelle über die logarithmischen \emph{Dezibel-Skala} ($dB$) angegeben \cite{tsuji_physics_2021, parker_good_2009}.

Ein Klang wird als \emph{Rauschen} bezeichnet, wenn es ein kontinuierliches Frequenzspektrum aufweist, das viele Arten von Geräuschen aus verschiedenen Quellen umfasst \cite{tsuji_physics_2021}.

Die Untersuchung des Aufbau eines Klang beschäftigt sich mit der Relation und Zusammenhang der verschiedenen Töne, die den Klang aufbauen. Die simpelste Form eines Tones ist der Sinuston, dessen Schwingung durch eine Sinuskurve beschrieben wird. Alle anderen Töne bauen sich, gemäß Fouriertheorems durch verschiedene Sinustöne zusammen, wobei die tiefste dominante Frequenz als \emph{Grundton} bezeichnet wird, und die höherer Frequenzen als \emph{Obertöne}. Ist die Frequenz eines Obertones ein vielfaches des Grundtones, wird dieser als \emph{harmonischer} Oberton bezeichnet. In dem Fall, dass der Oberton kein Vielfaches des Grundtones ist, sprechen wir von einem \emph{inharmonischen} Oberton. Wenn somit die gleiche Note an unterschiedlichen Instrumenten gespielt wird haben diese die gleiche Grundschwingung, jedoch unterschiedliche harmonische und inharmonische Obertöne \cite{parker_good_2009, white_physics_2014}.

Über ein Spektogramm kann die Zusammensetzung der Töne eines Klanges und die daraus resultierende Klangfarbe visualisieren werden. Über eine Fouriertransformation werden die Amplituden der verschiedenen Töne 
\subsection{Digitale Audiorepräsentation und Sampling}
\subsection{Synthesizer}

\section{Synthetisierung mittels Diffusion}
\subsection{Latente Diffsuion}
\subsection{AudioLDM}
\subsection{Finetuning}
\section{Implementierung des Neuronalen Synthesizers}
\subsection{Audioprogrammierung}
\subsection{JUCE Framework}
\glqq\emph{JUCE} ist das am häufigsten verwendete Framework für die Entwicklung von Audioanwendungen und -Plugins. Es handelt sich dabei um eine Open-Source-C++-Codebasis, die zur Erstellung eigenständiger Software auf Windows, macOS, Linux, iOS und Android sowie VST-, VST3-, AU-, AUv3-, AAX- und LV2-Plugins verwendet werden kann.\grqq \cite{noauthor_juce_nodate}

Es bietet eine Abstraktion für die Verarbeitung von Audiosamples und MIDI von den nativen Audiogeräten auf jeder Plattform oder einer Host-DAW. Mit der Bibliothek von \emph{digitalen Signalverarbeitungs-(DSP)-Bausteinen}, die JUCE bereitstellt, können unterschiedliche Audioeffekte, Filter, Instrumente und Generatoren schnell prototypisiert und eingesetzt werden. \cite{noauthor_juce_nodate} So umfasst wa eine breite Palette von Klassen, die häufig auftretende Probleme bei der Entwicklung von Audioprojekten lösen. Dazu gehören die Behandlung von Grafiken, Sound, Benutzerinteraktion und Netzwerken. \cite{robinson_getting_2013}

In dieser Arbeit wird das JUCE-Framework mittels \emph{CMake}, \glqq eine Open-Source-, plattformübergreifende Werkzeugfamilie, die zur Erstellung, zum Testen und zum Verpacken von Software entwickelt wurde\grqq \cite{noauthor_cmake_nodate} benutzt, um die durch das Diffusionsnetz erstellten Klänge spielbar und manipulierbar zu gestalten. 
\subsection{ONNX}
\subsection{Design}

\chapter{Ergebnisse}

\chapter{Diskussion}

\chapter{Schlussfolgerung}
\label{sec:conclusion}


\printbibliography

All links were last followed on \today{}.

\appendix
\input{latexhints/latexhints-english}

\pagestyle{empty}
\renewcommand*{\chapterpagestyle}{empty}
\Affirmation
\end{document}
